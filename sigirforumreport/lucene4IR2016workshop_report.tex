\documentclass[12pt]{article}
\usepackage{fancyhdr}
\usepackage{url}
\usepackage{color}
\usepackage{mathtools}

 \topmargin -0.8cm
 \oddsidemargin -0.7cm

 \textwidth 17.5cm
 \textheight 22.6 cm


\pagestyle{fancy} {\fancyhead{}
\fancyfoot[c]{\small{\rule{17.5cm}{1pt}}}}

\begin{document}
\title{\vspace{-2.5cm}
\begin{center}
\textbf{\small{Lucene4IR Workshop Report}}\\\vspace{-0.5cm} \rule{17.5cm}{1pt}
\end{center}
\vspace{1cm}\textbf{Report on the Lucene4IR workshop: Developing Information Retrieval Evaluation Resources using Lucene (L4IR2016) }}

\newcommand{\todo}[1]{\textcolor{red}{#1}}
\author{
Leif Azzopardi$^{1}$, Yashar Mosfeghi$^{2}$, Martin Halvey$^{1}$, \\
Krisztian Balog$^{3}$, Emanuele Di Buccio $^4$, Juan Manual Fernandez Luna $^{5}$,\\
 Charlie Hull$^{6}$, Jake Mannix$^{7}$, Sauparna Palchowdhury$^{8}$\\
    $^{1}$ {\small University of Strathclyde  \emph{ \{Leif.Azzopardi,Martin.Halvey\}@strath.ac.uk}}\\
    $^{2}$ {\small University of Glasgow\emph{\small Yashar.Mosfeghi@glasgow.ac.uk}}\\
	$^{3}$ {\small University of Stavanger \emph{\small krisztian.balog@uis.no}}\\
	$^{4}$ {\small University of Padova \emph{\small dibuccio@dei.unipd.it}}\\
	$^{5}$ {\small University of Granda \emph{\small jmfluna@decsai.ugr.es}}\\
	$^{6}$ {\small Flax \emph{\small charlie@flax.co.uk}}\\
	$^{7}$ {\small LucidWorks \emph{\small jake.mannix@lucidworks.com}}\\
	$^{8}$ {\small NIST \emph{\small sauparna.palchowdhury@nist.gov}}
}

\begin{sloppypar}

\maketitle \thispagestyle{fancy} 
\abstract{
The workshop and hackathon on developing Information Retrieval Evaluation Resources using Lucene (L4IR) was held on the 8th and 9th of September, 2016 at the University of Strathclyde in Glasgow, UK and funded by the ESF Elias Network. The event featured three main elements: (i) a series of keynote and invited talks on Lucene in action in industry, in teaching and learning environments, and evaluation forums. (ii) planning, coding and hacking where a number of groups created modules and infrastructure to use Lucene to undertake TREC based evaluations. And (iii) a number of breakout groups discussing challenges, opportunities and problems in bridging the divide between academia and industry, and how we can use Lucene and the resources created in teaching and learning IR evaluation. The event was composed of a mix and blend of academics, experts and students wanting to learn, share and create evaluation resources for the community. The hacking was intense and the discussions lively creating the basis of many useful tools and raising numerous issues. However, by adopting and contributing to most widely used and supported Open Source IR toolkit, it was clear that there were many benefits for academics, students, researchers, developers and practitioners - providing a basis for stronger evaluation practices, increased reproducibility, more efficient knowledge transfer, greater collaboration between academia and industry, and shared teaching and training resources.}

\maketitle



\section{Introduction}
Lucene and its expansions, Solr and ElasticSearch, represent the major open source Information Retrieval toolkits used in Industry. However, there is a lack of coherent and coordinated documentation that explains from an experimentalist's point of view how to use Lucene to undertake and perform Information Retrieval Research and Evaluation. In particularly, how to undertake and perform TREC based evaluations using Lucene. Consequently, the objective of this event was to bring together researchers and developers to create a set of evaluation resources showing how to use Lucene to perform typical IR operations (i.e.  indexing, retrieval, evaluation, analysis, etc.) as well as how to extend, modify and work with Lucene to extract typical statistics, implement typical retrieval models. Over the course of the workshop participants shared their knowledge with each other creating a number of resources and guides along with a road map for future development.


%!TEX root = lucene4IR2016workshop_report.tex
\section{Keynotes and Invited Talks}
%<<<<<<< HEAD
%During the course of the workshops a series of talks were given on how Lucene is being used in industry, teaching and for evaluation along with more technical talks on the inner workings of Lucene's scoring algorithm as well as how learning to rank is being included into Solr\footnote{\scriptsize{Slides are available from \url{www.github.com/leifos/lucene4ir}}}. 
%
%\subsection*{Introduction Talk: Why are we here?}
%{\bf Leif Azzopardi, University of Strathclyde}:
%Leif explained how after attending the lively Reproducibility workshop~\cite{arguello2016repro} at ACM SIGIR 2015, he wondered where the Lucene team was, and if Lucene and the community is so big, why they don't come to IR conferences - he posited that perhaps we haven't been inclusive and welcoming as we could to such a large community of search practitioners. He further asserted that this has lead to a failure in transferring our knowledge into one of the largest open source toolkits available. He argued that if we as academics want to increase our impact then we need to improve how we transfer our knowledge to industry. One way is working with large search engines, but what about other industries and organisations that need search and use toolkits like Lucene? He argued that we need to start speaking the same language i.e. work with Lucene et al. and look for opportunities on how we can contribute and develop resources for training and teaching IR and how to undertake evaluations and data science using widely used, supported and commonly accepted Open Source toolkits. He described how this workshop was a good starting point and opportunity to explore how academia and industry can better work together, where we can identify common goals, needs and resources that are needed to foster this relationship. 
%=======
During the course of the workshops a series of talks on how Lucene is being used in Industry, Teaching and for Evaluation along with more technical talks on the inner workings of how Lucene's scoring algorithm works and how learning to rank is being included into Solr, were presented\footnote{\scriptsize{Slides are available from \url{www.github.com/leifos/lucene4ir}}}. A summary of each talk is below.

\subsection*{Introduction Talk: Why are we here?}
{\bf Leif Azzopardi, University of Strathclyde}:
Leif explained how after attending the lively Reproducibility workshop~\cite{arguello2016repro} at ACM SIGIR 2015, he wondered where the Lucene team was, and why, if Lucene and the community is so big, why they don't come to IR conferences - he posited that perhaps we haven't been very inclusive or welcoming to such a large community of search practitioners. He further asserted that this has reduced our capacity to transfer our knowledge and experience into one of the largest Open Source toolkits available. He argued that if we as academics want to increase our impact then we need to improve how we transfer our knowledge to industry. One way is working with large search engines, but what about other industries and organisations that need search and use toolkits like Lucene? He argued that we need to start speaking the same language i.e. work with Lucene et al and look for opportunities on how we can contribute and develop resources for training and teaching IR and how to undertake evaluations and data science using widely used, supported and commonly accepted Open Source toolkits. He described how this workshop was a good starting point and opportunity to explore how academia and industry can better work together, where we can identify common goals, needs and resources that are needed to foster this relationship. 
%>>>>>>> f0e54b9adc801bc5760e93a856bf677df92c95d5

%!TEX root = Lucene4IR2016workshop_report.tex
\subsection*{Keynote Talk: Apache Lucene in Industry} 
{\bf Charlie Hull, Flax}: In his talk, Charlie first introduced Flax, and how it evolved over the years. Charlie explained that they have been building search applications using open search software since 2001. Their focus is on building, tuning and supporting fast, accurate and highly scalable search, analytics and Big Data applications. They are partners with Lucidworks, leading Lucene specialists and committers. When Lucene first came out clients were reluctant to adopt open source, but nowadays it is much more acceptable. Charlie notes that now you don't have to explain to clients what open source software is, and why it should be used. He described how Lucene-based search engines have risen in use - and that search and data analytics are available to those without six figure budgets. Charlie points out that Lucene is appealing because it is the most widely used open source search engine, which is hugely flexible, feature rich, scalable and performant. It is supported by a large and healthy community and backed by the Apache Software Foundation. Many of world's largest companies use Lucene including Sony, Siemens, Tesco, Cisco, Linkedin, Wikipedia, WordPress and Hortonworks. Charlie notes that they typically don't use Lucene directly, instead they use the search servers, built on top of Lucene, i.e. Apache Solr (which is mature, stable, and crucially highly scalable), or ElasticSearch (easy to get started with, great analytics, scalable). He contrasts these products with some of the existing toolkits in IR~\cite{Dowie2013,macdonald2012puppy,ogilvie2001experiments,zobel2004zettair}, and remarks on the latter, that ``no one in industry has ever heard of them!''. So even though they have the latest research encoded within them, it is not really viable for businesses to adopt them, especially as support for such toolkits is highly limited. He recommends that IR research needs to be within Lucene-based search services for it to be used and adopted. 

%Charlie then described a number of projects that they have been working on. (1) and (2)..

Based on Charlie's experience he provided us with a number of home truths:
\begin{itemize}
	\item Open source does not mean cheap 
	\item Most search engines are the same (in terms of underlying features and capabilities)
	\item Complex features are seldom used - and often confusing
	\item Search testing is rarely comprehensive
	\item Good search developers are hard to find
\end{itemize}


Charlie reflected on these points considering how we can do better. First, learn what works in industry and how industry are using search - there are lots of research challenges which they rarely get to solve and address but solutions to such problems would have real practical value. Second, improve Lucene et al with ideas from academia - faster - for example, it took years before BM25 replaced TFIDF as the standard ranking algorithm, where as toolkits like Terrier \cite{Ounis:2005:TIR:2149960.2150009} already have infrastructure for Learning to Rank, while this is only just being developed in Lucene. Third, he pointed out that testing and evaluation of Lucene based search engines is very limited, and that thorough evaluations by search developers is poor. He argued that this could be greatly improved, if academics and researchers, contributed to the development of evaluation infrastructure, and transferred their knowledge to practitioners on how to evaluate. Lastly, he pointed that the lack of skilled and knowledgable search developers was problematic - having experience with Lucene, Solr and ElasticSearch are highly marketable skills, especially, when there is a growing need to process larger and larger volumes of data - big data requires data scientists! So there is the pressing need to create educational resources and training material for both students and developers. 


%!TEX root = lucene4IR2016workshop_report.tex
\subsection*{Using Lucene for Teaching and Learning IR: The 
University of Granada case of study ?} 
{\bf  Prof. Juan Manual Fernandez Luna (University of Granda) }:
In this talk, we shall describe how the University of Granada is 
supporting teaching and learning Information Retrieval (TLIR) discipline 
across different courses in the Computer Science studies (degree and 
masters), and how this is done by means of the Lucene API. Later we 
shall present our thoughts about how Lucene could be used inTLIR context 
and present some proposals for improving the Lucene experience, both for 
students and lecturers.

%!TEX root = lucene4IR2016workshop_report.tex
\subsection*{Evaluation and Reproducible Experiments}
{\bf Sauparna ``Rup'' Palchowdhury (NIST) }:
Having seen students and practitioners in the IR community grapple with abstruse documentation that accompany search systems, I want to direct some attention to Lucene's internals and demonstrate how to do IR experiments using it. Ensuring that a search system you have built works ``correctly'' entails evaluating its output on test-collections like those from TREC. In this way evaluation finds a purpose in my exposition. The goal is to help bring Lucene to the IR community and prevent its usage as a black box, which misleads students because they learn little from their results. On top of that it makes it hard to reproduce experiments reported in papers. The talk will also go over Lucene's scoring by showing how a TFxIDF term-weighting scheme is implemented. I intend to initiate a discussion on implementing models of similarity and accept feedback to validate the implementation I describe.

To this end, I have built tools and written notes to help the experimenter organize her work: http://kak.tx0.org/IR/. Here is a
glorified Python script called TRECBOX that runs other search engines, notes on Terrier (TTR) and Lucene's (LTR) internals. LTR is a 'mod' that works on TREC data to help in quickly getting a system up and running; tests prove it to behave correctly on TREC test-collections.

%!TEX root = lucene4IR2016workshop_report.tex
\subsection*{ Deep Dive into the Lucene Query/Weight/Scorer Java Classes}
{\bf Jake Mannix, Lucidworks}:
In this more technical talk, Jake explained how Lucene scores a query, and what classes are instantiated to support the scoring. Jake described, first, at a high level how to do scoring modification to Lucene-based systems, including some ``Google''-like questions on how to score efficiently. Then, he went into more details about the BooleanQuery class and is cousins, showing where the Lucene API allows for modifications of scoring with pluggable Similarity metrics and even deep inner-loop, where ML-trained ranking models could be instantiated - \emph{if you're willing to do a little work}.


%!TEX root = lucene4IR2016workshop_report.tex
\subsection*{Learning to Rank with Solr} 
{\bf Diego Ceccarelli, Bloomberg}
On day two of the workshop, Diego started his talk by explaining that tuning Lucene/Solr et al is often performed by ``experts'' who hand tune and craft the weightings used for the different retrieval features. However, this approach is manual, expensive to maintain, and based on intuitive, rather than data. His working goal behind this project was to automate the process. He described how this motivated the use of Learning To Rank, a technique that enables the automatic tuning of a information retrieval system by applying machine learning when estimating parameters. He points out that sophisticated models can make more nuanced ranking decisions than a traditional ranking function when tuned in such a manner. During his talk, Diego presented the key concepts of Learning to Rank, how to evaluate the quality of the search in a production service, and then how the Solr plugin works. At Bloomberg, they have integrated a learning to rank component directly into Solr (and released the code as Open Source), enabling others to easily build their own Learning To Rank systems and access the rich matching features readily available in Solr. 





\section{Discussion}
During the course of the workshop, two breakout groups were formed to discuss how we can use Lucene when teaching and learning, and what were the main challenges in bridging the industry/academia along with what opportunities it could bring about. Finally, we asked participants to provide some feedback on the event.



%!TEX root = lucene4IR2016workshop_report.tex
\subsection{Teaching and Learning}
\label{sec:teaching}


\noindent \todo{Juanma}

\noindent \todo{Krisztian}

\noindent \todo{Martin}


How do you go about teaching IR? What level?

What kinds of things do you need/want from such resources?

How do we see Lucene fitting in? Benefits to students?

\noindent \todo{Emanuele}

As part of the discussion the experience gained in several courses was reported.

One concerned with the course of {\em Information Systems (Advanced)} of the Master Degree in Statistical Science at the University of Padua.\footnote{The description refers to the course editions in the Academic Years 2011/12-2014/15. The teacher in charge was Massimo Melucci.}
The course covered both basic IR topics -- e.g. indexing and retrieval methods, retrieval models, and evaluation -- and more advanced topics such as Web Search or Machine Learning for IR; a detailed description of the course contents can be found in~\cite{Melucci2013}, an IR book that actually stemmed from the experience gained during the diverse course editions.
The course was designed so that, for most of the topics, lessons at a theoretical level on a specific topic were followed by a laboratory assignment on that topic.
The topics covered in laboratory assignments were: (i) creation of a test collection, (ii) indexing, (iii) retrieval, (iv) relevance feedback, (v) link analysis, (vi) Learning to Rank or Optimization of Ranking functions with parameters. Students were asked to propose their own methodology to carry out the laboratory activity: for instance, when considering the topic of ``relevance feedback'', each student could propose its own methodology to perform feedback, e.g. through a query expansion method or term re-weighting.
Each assignment involved the experimental evaluation on a shared test collection. Indeed, the objective of the assignments was twofold: first, to better understand the topic; second, to become familiar in the design and the implementation of experimental methodologies to evaluate methods and/or components and, more in general, to test research hypotheses.
Students were allowed to use a manual approach (when possible), a software library or build their own software modules to achieve the assignment objective; a list of software libraries were provided before the first laboratory assignment to make the students aware of possible options.
The adoption of a manual approach for some of the laboratory activities was mandatory. For instance, in the case of the assignment on indexing, the use of a manual approach aimed at a better understanding of the conceptual mechanisms to identify the most effective descriptors to retrieve relevant documents.
When the students proposed their own methodology for indexing, they were asked to present their approach as a set of steps that can be automated.
The availability of software libraries or resources to easily use the basic operations is crucial to allow the students to test their methodology with little effort; Apache Lucene is an example of such libraries if complemented with examples or resources.
In an edition of the course, a lesson was dedicated to a general introduction to Apache Lucene; sample code were also provided. Along with Apache Lucene also elasticsearch~\cite{elasticsearch} was presented, particularly how to index documents, perform retrieval, and how to customize the scoring mechanism via scripting.\footnote{Elasticsearch allows to evaluate a custom score via scripts --- see the {\em Scripting} module. Apache Solr provides similar functionalities via {\em Function Queries}.} The main reason for the introduction to elasticsearch was that the students could index, retrieve, and customize retrieval -- and therefore test some of their methodologies -- without writing actual code but only through the use of REST requests.
The course was attended not only by students in Statistical Science, but also from other degree courses such as Computer Science or Computer Engineering.
The heterogeneous background of the students was one of the reasons for not restricting the laboratory activities to a single software library and to allow also usage of libraries, e.g. elasticsearch, where interaction is possible at a ``higher'' level.
While students in Computer Science and Computer Engineering were familiar with Java, students in Statistical Science tended to prefer the R language because it was used in many courses within their course degree. Therefore, one of the resources that could be useful for teaching is a R wrapper for Apache Lucene --- wrappers in other programming languages exist, e.g. PyLucene~\cite{PyLucene} for Python.
Software libraries such as elasticsearch, could be useful tools to support teaching: for instance, they provide functionalities -- in the event of elasticsearch a REST request -- to display how a specific fragment of text is processed given a pipeline, e.g. a specific tokenizer and a set of filters (lowercase, porter stemming, $\dots$).


\begin{table}[]
  \small
\centering
\caption{Attempt at encoding the picture}
\label{my-label}
\begin{tabular}{|l|l|l|}
\hline
Apps                    & High Level                                                                                                            & Low Level                                                                                                                                                           \\ \hline
IndexerApp              & \begin{tabular}[c]{@{}l@{}}Modify how the indexer is performed\\ i.e. different tokenizers, parsers, etc\end{tabular} & Can modify parsers, tokenizers, etc                                                                                                                                 \\ \hline
IndexAnalyzerApp        & Inspect the influence of indexer                                                                                      &                                                                                                                                                                     \\ \hline
RetrievalApp            & \begin{tabular}[c]{@{}l@{}}Try out different retrieval algorithms\\ Change retrieval parameters\end{tabular}          & Implement new retrieval algorithms                                                                                                                                  \\ \hline
trec\_eval              & Measure the performance                                                                                               &                                                                                                                                                                     \\ \hline
ResultAnalyzerApp       & Inspect and analyze the results returned                                                                              & \begin{tabular}[c]{@{}l@{}}Customise the analysis, put out other \\ statistics of interest\end{tabular}                                                             \\ \hline
ExampleApp              &                                                                                                                       & \begin{tabular}[c]{@{}l@{}}Examples of how to work with the Lucene\\ index, to make modifications\end{tabular}                                                      \\ \hline
Batch Retrieval Scripts & \begin{tabular}[c]{@{}l@{}}Configure to run a series of standard\\ batch experiments\end{tabular}                     & \begin{tabular}[c]{@{}l@{}}Customize to run specific retrieval\\ experiments\end{tabular}                                                                           \\ \hline
RetrievalShellApp       & n/a                                                                                                                   & \begin{tabular}[c]{@{}l@{}}Customize to implement retrieval algorithms \\ outwith the Lucene scorer i.e. a simple scorer \\ assuming term independence\end{tabular} \\ \hline
\end{tabular}
\end{table}

%!TEX root = lucene4IR2016workshop_report.tex

\subsection{Challenges and Opportunities}
During the workshop the challenges and opportunities between academia and industry were discussed, focusing on research, teaching and learning, and graduate attributes. Below is a summary of the main points stemming from the discussion.

The first point discussed were research opportunities and challenges that can arrises between industry and academia in information retrieval, text mining and big data domain. 
More specifically to the field of search and big data, academic participants felt that there was a lack of good and big real-world test data for researchers. 
They expected companies to share data with that quality that is representative enough of the current challenges companies facing. 
In turn, academia can provide solutions which can be feed back to these companies as a potential solution that can be deployed by them in the future.  
It was noted communicating current challenges or sharing data for companies might be difficult due to the Non-disclosure agreement (NDA) they have with their clients, at least this is the case for small open sourced companies. 
The industry participants mentioned that a scenario in which this situation can be realised is that a client come forward with a funding for investing in solving a challenging problem that requires an academic to do so, but this does not happen that often. 

The next point discussed were the need for a common open source platform that can be used for teaching and learning information retrieval and big data. 
For example, both academic and industry participants agreed upon the need for a better and general documentation for Lucene which is not bound to any specific features that is version dependent. 
The industry participants felt that such teaching and learning materials should be developed by academics. 
It was also noted the possibility of a collaboration between academia and industry via a funding scheme such as the Knowledge Transfer Partnerships\footnote{\scriptsize{\url{http://www.esrc.ac.uk/funding/funding-opportunities/knowledge-transfer-partnerships/}}}. 
However, the industry participants mentioned that they expected academia take the lead on this since they have the required expertise to apply for grant applications. 

The final point discussed were the graduate attributes and the skills required by Computer Scientist and Software Engineering Graduates working in information retrieval, text mining and big data. 
The obvious and core skills required in terms of being able to : (i) develop high quality, robust code, (ii) understand and think about complex systems/problems, (iii) communicate, discuss and resolve issues and (iv) have a good understanding of software engineering principles and practices, were seen as mandatory. 
More specifically to the field of search and big data, industry participants felt that there was a lack of skill graduates that knew about search technologies, how to process large scale data sets, and how to work with big (text) data. 
They expected candidates to understand more about the processes and core concepts of search and big data - to be able to describe it at a conceptual level, at least - but with ideally some practical skills (i.e. Lucene, Solr, Spark, Pig, etc). 
It was noted that the learning curve can be very steep when picking such technologies - but such skills are seen to be increasingly valuable by employers. 
For example, Lucene is a very large and complex project, that has evolved over years, so it can feel very opaque and daunting to begin with, however, given that it one of the largest OS toolkits for information and data mining and retrieval it is a skill worth learning. 
It was felt that there was a strong need for developing more training and resources for beginners to learn how to use such toolkits - and this was seen to be an area where more investment from funding agencies and universities could be directed. 






%In addition, many open source search companies are small and therefore they find it hard to support apprenticeships, internships, etc. 


%!TEX root = lucene4IR2016workshop_report.tex
\subsection{Feedback}


%!TEX root = lucene4IR2016workshop_report.tex
\section{Resources}
As part of the workshop numerous attendees contributed to the Lucene4IR GitHub Repository - \url{http://github.com/leifos/lucene4ir/}. In the repository, three main applications were developed and worked on:
\begin{itemize}
	\item IndexerApp - enables the indexing of several different TREC collections, e.g. TREC123 News Collections, Aquaint Collection, etc.
	\item RetrievalApp - a batch retrieval application when numerous retrieval algorithms can be configured, e.g. BM25, PL2, etc
	\item ExampleStatsApp - an application that shows how you can access various statistics about terms, documents and the collection. e.g. how to access the term posting list, how to access term positions in a document, etc.
	\end{itemize}
	
In the repository, a sample test collection (documents, queries and relevance judgements) was provided (CACM), so that participants could try out the different applications.

During the workshop, a number of different teams undertook various projects:
\begin{itemize}
	\item Customisation of the tokenisation, stemming and stopping during the indexing process: this enabled the IndexerApp to be configured so that the collections can be indexed in different ways - the idea being that students would be able to vary the indexing and then see the effect on performance.
	\item Implementation of other retrieval models: inheriting from Lucene's BM25Similiarity Class,  BM25 for Long documents was implemented BM25L\cite{}, OKAPI BM25's was also implemented to facilitate the comparison between how it is currently implemented in Lucene versus an implementation of the original BM25 weighting function~\cite{}.
	\item Rather than scoring through Lucene's mechanics, others attempted to implement BM25 by directly accessing the inverted index - again to provide a comparison in terms of both efficiency and effectiveness for scoring queries.
	\item A QueryExpansionRetrievalApp
	\item Hacking the innerloop
	\item Additional Examples on how to access and work with Lucene's index and 
\end{itemize}


The break-out group focused on inner loop scoring wanted to try something that was simultaneously simple, practical, and yet required some inner loop scoring magic.  Based on the interests of the group members, we decided on "cross-field phrase queries": an extension of the idea of a sloppy phrase query where the "slop" allowed for a pair of terms occurring in *different* fields to be part of a phrase (but with a parametrizably lower score than terms in the same field).  We worked out the design (delegating most of the work to Query / Weight / Scorer classes already in Lucene, but then combining them together across fields), and stepped through much of the iteration implementation.  While we got most of the "plumbing" done, we only had enough time for our "score()" method to be implemented as naively as imaginable, and did not get it fully working in the time of the workshop.  Some participants expressed interest in working on it further, to see how efficient it was, and what effect on scoring it would have (if a QueryParser was configured to explicitly spit out queries of this form sometimes).


\section{Summary}

\section{Acknowledgments}
We thank the European Science Foundation / ELIAS Network for funding the workshop (Grant No. SM 5916). We would also like to thank our speakers as well as Bloomberg, FlaxSearch, LucidWorks and the University of Strathclyde. Finally, we would like to thank all the participants for their contributions to the workshops and hackathon. Also, thanks to Manisha, Guido and Casper for their offline contributions.


\todo{Add references to indri, lemur, terrier, lucene, etc, BM25L}

\bibliography{lucene4IR}{}
\bibliographystyle{acm}

%\bibliography{refWS}
\end{sloppypar}
\end{document}
