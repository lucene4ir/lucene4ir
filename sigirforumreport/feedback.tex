%!TEX root = lucene4IR2016workshop_report.tex
\subsection{Feedback}

At the end of the workshop, we asked participants to provide feedback on the following questions: ``What would you suggest us to Stop/Start/Continue to do in our next workshop?''. Below is a summary of the points contributed by participants.

First of all, participants were pleased by the workshop, specially they enjoyed the hackathon part as well as the talks from the industry.  They also encouraged us to organise a follow up workshop on this topic. Participants liked the idea of the workshop were it brought together academics and industry and encouraged us to continue inviting people from industry and in particular Lucene developers. They also suggested to invite more undergraduate students to the hackathon.

Some participants suggested defining the goals and objectives by asking participants before the hackathon.  In addition to this point, other participants suggested that they would prefer to have a more well-defined structured hackathon, which means, setting more well-designed goals, having tools and data tested and ready, as well as making sure all participants use the same data.  One proposal on this front was to create a standardised mini-competition between workshop participants, e.g. providing a template code with a challenge to increase MAP with the expectation that each team formally presents its results.

Since, a number of the participants were not very familiar with using Lucene, it was suggested that in subsequent hackathons a mini-tutorial is included to help participants get up to speed. For example, some demos of various Lucene modules and how they work. Others suggested that some tutorials for the IR community would be a great way to encourage people to start working with Lucene - and in particular - help students to learn the basics and how to work with the toolkit, instead of against it. Some more detailed technical talks on the various modules were also encouraged.

Finally, some participants were requesting to include Solr into the program and promote the event in the Lucene and Solr Communities. And that the workshop could have been recorded and/or stream to increase its reach and dissemination. 

We are very encouraged by the suggestions and feedback from participants as they provide a number of ways in which this initiative can be further developed and improved to support research and industry in this area.
