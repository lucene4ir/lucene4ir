%!TEX root = lucene4IR2016workshop_report.tex
\subsection{Feedback}

At the end of the workshop we asked participants to answer the following question: "What would you suggest us to Stop/Start/Continue to do in our next workshop?". 
Here we present main points derived from this feedback. 
First of all, participants were pleased by the workshop, specially they enjoyed the hackathon part as well as the talks from the industry. 
They also encouraged us to organise a follow up workshop on this topic.   
%Continue	Hackathon
%Continue	Being Cool People
%Continue	Hackathon Style between/after talks
%Continue	Providing great food, fun and helpful experts/speakers.
%Continue	Everything else! Very good workshop
%Continue	Being awesome
%Continue	Giving generous grants
%Continue	The general idea of the workshop is perfect
%Continue	Build useful tools for teaching 
Participants liked the idea of the workshop were it brought together academics and industry and encouraged us to continue inviting people from industry and in particular Lucene developers.
They also suggested to Invite more undergraduate students to the talks and hackathon.  
%Start	Make the event a bit longer, 3 days maybe? 
%Start	Invite more undergraduate students to the talks/hackathons.
%Start	Get more involved in other areas of the research 
%Start	Would be cool to have 50% Academia, 50% Industry
%Continue	Talks by people in the industry 
%Continue	Bringing in people from the industry to share how they do things internally 
%Continue	Invite talks from industry
%Start	Invite some of Lucene's developer

In addition, participants suggested that they would prefer to have a more well-defined structured hackathon, which means, setting more well-designed goals, having tools and data tested and ready, as well as making sure all participants use the same data. 
Some suggested to define the goals and objectives by asking participants before the hackathon. 
%Start	More structure to the hackaton, which means, setting more well-designed goals, having tools & data tested & ready. Making sure all participantes use the same data.
%Stop	Not having a set of primary objectives, challenges
%Stop	Leave complete freedom about what to hack (not so productive/interesting)
%Start	Maybe let's try to define things to implemet before (also asking in the mailing list)
%Start	Clear objectives for the hacking part, maybe some fixed tasks beforehand?
%Start	think ahead experiments we can do during the workshop
One proposed idea was to create a standardised mini-competition between workshop participants, e.g. providing a template code with a challenge to increase MAP with the expectation that each team formally presents its results.
%Start	Creating standardised mini-competitions between workshop participants. E.g. provide some template code with a challenge to increase MAP
%Start	Formal presenation of the resutls obtained by the teams
%Start	Creating larger groups working along the time and meeting in the mext workshop
Another proposed idea was to organise a mini-tutorial on Lucene features (related to the hackathon), or provide active demos of Lucene module implementation. 
A related proposed idea was to provide a basic quick-start workshop and documentation for people unfamiliar with Lucene. 
%Start	Ask one of the users to do a mini-tutorials on one of the Lucene features (Maybe related to the Hackathon)
%Start	Providing a basic quick-start workshop for people unfamiliar with Lucene 
%Start	Write documentation for people that do not know Lucene
%Start	Beginners talk on the fist day for those unfamiliar with Lucene 
%Start	Organise tutorial on Lucene/Solr, tape them and put on Youtube
%Start	Active demos of implementing moduler
Participants also requested for shorter talks as well as recording and streaming them. 
%Stop	Long talks. Cap it to 20 min max
%Start	Recording, streaming the talks
%Start	Have the break-out groups produce some written progress report along the way (Readme file on Github)
They also requested to have slightly more technical talks focusing on Lucene architecture. 
%Stop	Talk could be a bit more technical focusing on Lucene architecture
Finally, some participants were requesting to include Solr into the program and promote the event in the Lucene and Solr Communities.
%Start	Using More Solr
%Start	Doing groups some hacking Lucene and some do higher level Solr Stuff
%Start	Promote more in the Lucene/Solr /Elastico Community


%There were few suggestions with regards to what they would like us to stop doing in our next workshop. 
%The most emphasised one was the use of the projector screen. 
%This is due to the fact that the projector screen was a whiteboard with two grooves running down its face which have made seeing the contents of the slides very hard. 
%We were limited to the facilities we had at that time, and we are looking into alternative options for our future venue. 
%%Stop	Using a splot white board with two grooves runnig down its face for presentation. It was very hard to see the content of the slides. 
