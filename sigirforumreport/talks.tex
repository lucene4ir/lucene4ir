%!TEX root = lucene4IR2016workshop_report.tex
\section{Keynotes and Invited Talks}
During the course of the workshops a series of talks were given on how Lucene is being used in industry, teaching and for evaluation along with more technical talks on the inner workings of Lucene's scoring algorithm as well as how learning to rank is being included into Solr\footnote{\scriptsize{Slides are available from \url{www.github.com/leifos/lucene4ir}}}. 

\subsection*{Introduction Talk: Why are we here?}
{\bf Leif Azzopardi, University of Strathclyde}:
Leif explained how after attending the lively Reproducibility workshop~\cite{arguello2016repro} at ACM SIGIR 2015, he wondered where the Lucene team was, and if Lucene and the community is so big, why they don't come to IR conferences - he posited that perhaps we haven't been inclusive and welcoming as we could to such a large community of search practitioners. He further asserted that this has lead to a failure in transferring our knowledge into one of the largest open source toolkits available. He argued that if we as academics want to increase our impact then we need to improve how we transfer our knowledge to industry. One way is working with large search engines, but what about other industries and organisations that need search and use toolkits like Lucene? He argued that we need to start speaking the same language i.e. work with Lucene et al. and look for opportunities on how we can contribute and develop resources for training and teaching IR and how to undertake evaluations and data science using widely used, supported and commonly accepted Open Source toolkits. He described how this workshop was a good starting point and opportunity to explore how academia and industry can better work together, where we can identify common goals, needs and resources that are needed to foster this relationship. 

%!TEX root = Lucene4IR2016workshop_report.tex
\subsection*{Keynote Talk: Apache Lucene in Industry} 
{\bf Charlie Hull, Flax}: In his talk, Charlie first introduced Flax, and how it evolved over the years. Charlie explained that they have been building search applications using open search software since 2001. Their focus is on building, tuning and supporting fast, accurate and highly scalable search, analytics and Big Data applications. They are partners with Lucidworks, leading Lucene specialists and committers. When Lucene first came out clients were reluctant to adopt open source, but nowadays it is much more acceptable. Charlie notes that now you don't have to explain to clients what open source software is, and why it should be used. He described how Lucene-based search engines have risen in use - and that search and data analytics are available to those without six figure budgets. Charlie points out that Lucene is appealing because it is the most widely used open source search engine, which is hugely flexible, feature rich, scalable and performant. It is supported by a large and healthy community and backed by the Apache Software Foundation. Many of world's largest companies use Lucene including Sony, Siemens, Tesco, Cisco, Linkedin, Wikipedia, WordPress and Hortonworks. Charlie notes that they typically don't use Lucene directly, instead they use the search servers, built on top of Lucene, i.e. Apache Solr (which is mature, stable, and crucially highly scalable), or ElasticSearch (easy to get started with, great analytics, scalable). He contrasts these products with some of the existing toolkits in IR~\cite{Dowie2013,macdonald2012puppy,ogilvie2001experiments,zobel2004zettair}, and remarks on the latter, that ``no one in industry has ever heard of them!''. So even though they have the latest research encoded within them, it is not really viable for businesses to adopt them, especially as support for such toolkits is highly limited. He recommends that IR research needs to be within Lucene-based search services for it to be used and adopted. 

%Charlie then described a number of projects that they have been working on. (1) and (2)..

Based on Charlie's experience he provided us with a number of home truths:
\begin{itemize}
	\item Open source does not mean cheap 
	\item Most search engines are the same (in terms of underlying features and capabilities)
	\item Complex features are seldom used - and often confusing
	\item Search testing is rarely comprehensive
	\item Good search developers are hard to find
\end{itemize}


Charlie reflected on these points considering how we can do better. First, learn what works in industry and how industry are using search - there are lots of research challenges which they rarely get to solve and address but solutions to such problems would have real practical value. Second, improve Lucene et al with ideas from academia - faster - for example, it took years before BM25 replaced TFIDF as the standard ranking algorithm, where as toolkits like Terrier \cite{Ounis:2005:TIR:2149960.2150009} already have infrastructure for Learning to Rank, while this is only just being developed in Lucene. Third, he pointed out that testing and evaluation of Lucene based search engines is very limited, and that thorough evaluations by search developers is poor. He argued that this could be greatly improved, if academics and researchers, contributed to the development of evaluation infrastructure, and transferred their knowledge to practitioners on how to evaluate. Lastly, he pointed that the lack of skilled and knowledgable search developers was problematic - having experience with Lucene, Solr and ElasticSearch are highly marketable skills, especially, when there is a growing need to process larger and larger volumes of data - big data requires data scientists! So there is the pressing need to create educational resources and training material for both students and developers. 


%!TEX root = lucene4IR2016workshop_report.tex
\subsection*{Using Lucene for Teaching and Learning IR: The 
University of Granada case of study ?} 
{\bf  Prof. Juan Manual Fernandez Luna (University of Granda) }:
In this talk, we shall describe how the University of Granada is 
supporting teaching and learning Information Retrieval (TLIR) discipline 
across different courses in the Computer Science studies (degree and 
masters), and how this is done by means of the Lucene API. Later we 
shall present our thoughts about how Lucene could be used inTLIR context 
and present some proposals for improving the Lucene experience, both for 
students and lecturers.

%!TEX root = lucene4IR2016workshop_report.tex
\subsection*{Evaluation and Reproducible Experiments}
{\bf Sauparna ``Rup'' Palchowdhury (NIST) }:
Having seen students and practitioners in the IR community grapple with abstruse documentation that accompany search systems, I want to direct some attention to Lucene's internals and demonstrate how to do IR experiments using it. Ensuring that a search system you have built works ``correctly'' entails evaluating its output on test-collections like those from TREC. In this way evaluation finds a purpose in my exposition. The goal is to help bring Lucene to the IR community and prevent its usage as a black box, which misleads students because they learn little from their results. On top of that it makes it hard to reproduce experiments reported in papers. The talk will also go over Lucene's scoring by showing how a TFxIDF term-weighting scheme is implemented. I intend to initiate a discussion on implementing models of similarity and accept feedback to validate the implementation I describe.

To this end, I have built tools and written notes to help the experimenter organize her work: http://kak.tx0.org/IR/. Here is a
glorified Python script called TRECBOX that runs other search engines, notes on Terrier (TTR) and Lucene's (LTR) internals. LTR is a 'mod' that works on TREC data to help in quickly getting a system up and running; tests prove it to behave correctly on TREC test-collections.

%!TEX root = lucene4IR2016workshop_report.tex
\subsection*{ Deep Dive into the Lucene Query/Weight/Scorer Java Classes}
{\bf Jake Mannix, Lucidworks}:
In this more technical talk, Jake explained how Lucene scores a query, and what classes are instantiated to support the scoring. Jake described, first, at a high level how to do scoring modification to Lucene-based systems, including some ``Google''-like questions on how to score efficiently. Then, he went into more details about the BooleanQuery class and is cousins, showing where the Lucene API allows for modifications of scoring with pluggable Similarity metrics and even deep inner-loop, where ML-trained ranking models could be instantiated - \emph{if you're willing to do a little work}.


%!TEX root = lucene4IR2016workshop_report.tex
\subsection*{Learning to Rank with Solr} 
{\bf Diego Ceccarelli, Bloomberg}
On day two of the workshop, Diego started his talk by explaining that tuning Lucene/Solr et al is often performed by ``experts'' who hand tune and craft the weightings used for the different retrieval features. However, this approach is manual, expensive to maintain, and based on intuitive, rather than data. His working goal behind this project was to automate the process. He described how this motivated the use of Learning To Rank, a technique that enables the automatic tuning of a information retrieval system by applying machine learning when estimating parameters. He points out that sophisticated models can make more nuanced ranking decisions than a traditional ranking function when tuned in such a manner. During his talk, Diego presented the key concepts of Learning to Rank, how to evaluate the quality of the search in a production service, and then how the Solr plugin works. At Bloomberg, they have integrated a learning to rank component directly into Solr (and released the code as Open Source), enabling others to easily build their own Learning To Rank systems and access the rich matching features readily available in Solr. 



